\subsection{Instalacja niezbędnych narzędzi}
Pierwszym krokiem było zainstalowanie Unity – silnika, na którym utworzona została nasza gra. 

Do tworzenia oraz edycji skryptów, używaliśmy dwóch środowisk, Visual Studio IDE oraz Visual Studio Code. 

Dla usprawnienia wspólnej pracy użyliśmy rozproszonego systemu kontroli wersji GIT. Korzystaliśmy zarówno z programu używając wiersza poleceń jak i aplikacji GitHub Desktop.

Podczas pracy nad naszym projektem często działaliśmy na tych samych, obszernych plikach np. w przypadku edycji mapy, po której poruszają się gracze. 
Przez to podczas łączenia naszych zmian często dochodziło do konfliktów w kodzie, których system kontroli wersji nie mógł sam rozwiązać. 
W tym wypadku niezbędne okazało się narzędzie UnityYAMLMerge, które znacznie usprawniło łączenie dwóch kopii, bez konieczności manualnego rozwiązywania konfliktów.